\documentclass[10pt,a4paper]{article}
\usepackage{listings}
\usepackage[utf8]{inputenc}
\usepackage{color}
\usepackage[francais]{babel}
\usepackage[T1]{fontenc}
\usepackage{graphicx}
\usepackage[export]{adjustbox}
\bibliographystyle{ieeetr}
\author{Ali CHERIFI, Clément FONTENAY, Valentin GAILLARD}
\title{Projet de compilation\\Intepréteur de Pseudo-Pascal \\Traducteur de Pseudo-Pascal vers C3A}
\begin{document}
\maketitle
\newpage
\tableofcontents
\newpage
\section{Introduction}

Le but de ce projet est de manipuler la théorie des langages via les outils Bison et Flex sur le langage Pseudo-Pascal.
Ce langage est assez complet puisqu'il gère le typage des variables (booléen et entier), les tableaux, les appels de fonctions ainsi que la portée des variables.
Nous avons pour cela à réaliser les programmes suivant :\\ 
\begin{itemize}
    \item Un interpréteur de Pseudo-Pascal.
    \item Un traducteur de Pseudo-Pascal vers C3A (code à 3 adresses).
    \item Un interpréteur de C3A.
\end{itemize}

\section{Cahier des charges}
\subsection{Besoins fonctionnels}

\begin{itemize}
    \item Prendre en entrée un programme Pseudo-Pascal au format .pp.
    \item Vérifier que le programme est correct syntaxiquement et sémantiquement.
    \item Interpréter ce programme pour en dégager le résultat.
    \item Traduire le programme en C3A.
    \item Interpréter ce programme C3A.
\end{itemize}

\subsection{Besoins non foncitonnels}

\begin{itemize}
    \item Indication sur les erreurs commises par l'utilisateur lors de l'écriture du programme Pseudo-Pascal.
    \item Traduction du programme relativement rapide.\\
\end{itemize}

Nous ne traiterons pas les options demandées c'est-à-dire le passage au langage assembleur Y86 et le ramasse-miettes.
Nous avons fait le choix de nous concentrer sur les fonctionnalités principales afin de les implémenter de la meilleure manière possible.

\subsection{Architecture}
Nous avons découpés notre travail de façon à créer plusieurs modules qui peuvent être assemblés par la suite.\\

\begin{itemize}
    \item Les fichiers .y et .l permettent la lecture du programme d'entrée et l'analyse syntaxique.
    \item Le module d'analyse sémantique se trouve dans les fichiers analyseur.c/.h
    \item L'interpréteur se trouve dans le fichier interpreteur.c
\end{itemize}



\section{Implémentation}
\subsection{Description globale}
Pour analyser et par la suite interpréter notre programme nous nous sommes reposer sur la lecture faite par Bison.
Bison pouvant générer un arbre du programme lu via sa grammaire, nous pouvons par la suite lire cet arbre par un parcours préfixe et donc l'interpréter.

Nous avons implémenté une structure de noeud comportant les champs codop (Code de l'opération en cours), \textit{ETIQ} (Nom de variable, valeur, etc..), \textit{FG}, \textit{FD} 
(respectivement fils gauche et fils droit). Dans notre grammaire les commandes et expressions sont de type noeud ce qui nous permet de construire l'arbre du programme.

Nous disposons également de notre propre structure \textit{EnvG} qui regroupe la totalité du programme lu via un \textit{BILENV} stockant les variables globales, un noeud 
racine comportant tout le programme principal (sans les déclarations de fonctions auxiliaires) et une liste \textit{BILFON} répertoriant les déclarations de fonctions. 
Le non-terminal MP est de type \textit{EnvG}, l'interpréteur (ou le traducteur) à alors accès à tous les éléments composant un programme.

\subsection{Description de l'analyseur sémantique}
Tout d'abord pour chaque utilisation d'une variable, nous regardons si elle est définie. Pour ça, nous recherchons dans \textit{ListeVariablesLOCALES} puis dans \textit{ListeVariablesGLOBALES}. Nous remplissons \textit{ListeVariablesLocales} à chaque fois fois qu'il y a une définition de fonction pour y mettre dedans les paramètres de la fonction; puis nous rajoutons à la liste les variables déclarées dans la fonction.
Après cette vérification, nous vérifions que les types sont cohérents, pour les affectations, les appels de fonctions, les opérations ...Et nous finissons par afficher le message adéquat et la ligne.

\subsection{Problèmes rencontrés et solutions}
\subsubsection {Pour l'analyseur syntaxique}
Le premier problème était, les conflits générés par la grammaire; nous les avons régler un par un en analysant le bison avec la commande JESAISPASDEDEDEDEDED , mais pour pouvoir un conflit dans C avec Se, nous avons du changer C et rajouter dans la grammaire Ca:\newline 
C: C Se C

	| Sk
    
	|'{' C '}'
    
	| V '(' L\_args ')'
    
	| Ca\newline\newline
Ca: Wh E Do C

	| If E Th C El Ca
    
	| Et Af E
    
	| V Af E

Le second problème était de récupérer le nom des variables, donc pour ça nous avons d'abord essayer en rajoutant simplement, dans l'union yylval, un char* nom, mais n'étant pas la bonne solution car en construisant l'arbre avec bison, yylval.nom n'avait pas la bonne valeur au bon moment. Donc pour solutionner ce problème nous avons rendu V terminal, pour ça nous avons rajouté yylval.noeud; et c'est dans le flex, au moment de lire une variable, que nous créons un nouveau noeud et que nous mettons dans yylval.noeud.

Le dernier problème, qui était sans doute le plus petit, était de savoir, dans le flex, si on a une variable, une fonction ou une procedure; donc pour ça nous avons ajouté une variable dans le flex que nous modifions au moment de lire \textit{defun} ou \textit{defpro}.
\subsubsection {Pour l'analyseur sémantique}
-tableau\newline
-le retour des fonctions\newline


\section{Phase de test}


\newpage
\section{Conclusion}


\end{document}
